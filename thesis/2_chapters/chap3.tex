% Chapter 3
\chapter{Hardware and Software Co-processing – Implementation of the Algorithm}
\label{chap:HSC}

%%%%%%%%%%%%%%%%%%%%%%%%%%%%%%%%%%%%%%%%%%%%%%%%%%%%%%%%%%%%%%%%%%%%%%%%%%%%%%%%%%%%%%%%%%%%%%%%%%%%%%%%%%%%%%%%%%%%%%%%%%%%%%%%
% Section 3.1
\section{Overview}

% Subsection 3.1.2
\subsection{FPGA - Programmable Logic - IP Block}

FPGA (Field Programmable Gate Array) provides a general purpose digital logic with great flexibility and utility. The fundamental logic cells of an FPGA work using look up tables (LUTs). The performance of an FPGA design depends on the area using and routing of LUTs\cite{Dummies}. \\

\noindent \textbf{LUTs}: FPGAs were first designed to imitate Gate Arrays – integrated circuits with a sea of gates that were connected by added metal layers (wire routing) to define the functionality of the device. However, instead of implementing the logic using transistor gates, the logic was in many cases implemented by memory fashioned as look up tables. \\

\ntuepsfig[width=\textwidth]{3_LUT}{A 4-input logic implemented by a Look up table (LUT)\cite{Coursera}}

\noindent \textbf{Routing}: The routing in an FPGA is hierarchical. At the lowest level, we have a routing switch that is controlled by a bit of SRAM, or a FLASH bit, or a fuse. There are many routing switches in the local routing to connect various logic elements or blocks. The FPGA is made of a sea of these logic blocks. Groups of the blocks are connected by long lines or very long lines. Common signals like clocks are routed on special routing networks called global networks. \\

\ntuepsfig[width=\textwidth]{3_Routing}{FPGA Routing Architecture\cite{Coursera}}

\noindent \textbf{FPGA design flow}: The design begins with design entry, continues with simulation to verify the logic is implemented as expected, and then the tool will perform synthesis (also called mapping) to map the logic to the device architecture. Next the tool will create the interconnection between cells by place and routing (or fitting) the design. Another simulation after the fitting is done is a good practice. If this looks good, the next step to use the tool to create a programming file, which is then downloaded into the FPGA device for testing\cite{Coursera}. \\

\ntuepsfig[width=\textwidth]{3_DesignFlow}{FPGA Development Process Design Flow\cite{Coursera}}

\noindent \textbf{Programmable Logic}: In Xilinx architecture, the FPGA part of the design is called Programmable Logic (PL). Some of the specialized components in the architecture include:
\begin{itemize}
	\item \textbf{Configurable Logic Blocks (CLBs)} contain flexible Look-Up Tables (LUTs) that implement logic plus storage elements used as flip-flops or latches.
	\item \textbf{Input/Output Blocks (IOBs)} control the flow of data between the I/O pins and the internal logic of the device. IOBs support bidirectional data flow plus 3-state operation. They support a variety of signal standards, including several high-performance differential standards. Double Data-Rate (DDR) registers are included.
	\item \textbf{Block RAM} provides data storage in the form of 18-Kbit dual-port blocks.
	\item \textbf{Multiplier Blocks} accept two 18-bit binary numbers as inputs and calculate the product.
	\item \textbf{Digital Clock Manager (DCM) Blocks} provide self-calibrating, fully digital solutions for distributing, delaying, multiplying, dividing, and phase-shifting clock signals.
\end{itemize}

\noindent These elements are organized as shown in Figure~\ref{fig:3_spartan}. A dual ring of staggered IOBs surrounds a regular array of CLBs. Each device has two columns of block RAM. Each RAM column consists of several 18-Kbit RAM blocks. Each block RAM is associated with a dedicated multiplier. The DCMs are positioned in the center with two at the top and two at the bottom of the device. The design has two DCMs in the middle of the two columns of block RAM and multipliers. \\

\noindent The Xilinx Programmable Logic features a rich network of traces that interconnect all five functional elements, transmitting signals among them. Each functional element has an associated switch matrix that permits multiple connections to the routing\cite{spartan}. \\

\ntuepsfig[width=\textwidth]{3_spartan}{Xilinx Spartan-3AN Family Architecture\cite{spartan}}

\noindent \textbf{Intellectual property core}, IP core, or IP block is a reusable unit of logic, cell, or integrated circuit (commonly called a "chip") layout design that is the intellectual property of one party. IP cores may be licensed to another party or can be owned and used by a single party alone. The term is derived from the licensing of the patent and/or source code copyright that exist in the design. IP cores can be used as building blocks within application-specific integrated circuit (ASIC) designs or field-programmable gate array (FPGA) logic designs\cite{ipcore}.

\pagebreak 
\subsection{System-on-a-chip}

System-on-a-chip (SoC) is anatomy in embedded system which considers a processing board integrated with processor and hardware peripherals and memory for a specific function like digital, analog signal processing\cite{12457872420171101}. \\

\noindent Traditional SoC board is a fixed Application Specific Integrated Circuit (ASIC) with is prebuilt for only one particular function or task. The processing time of ASIC board is speedy and high-efficiency due to the unique hardware prebuilt for this task. In the trade-off, the development time of an ASIC board takes much time due to multiple stages of hardware synthesis, testing and verification and the cost of development is enormous. ASIC board is only suitable for mass-production strategy with over 1 million products per a development cycle. \\

% Figure 3.1
\ntuepsfig[width=\textwidth]{3_1}{An ASIC board}

\noindent Another approach is using an embedded processor for the implementation. All stages of the development are equivalent to a software development process. This approach brings some advantages like low-cost development, flexibility, easy to fix and testing, short-time-development, and short-time-to-market. On the downside, the approach of using multipurpose processors causes high power consumption, low efficiency in the product operation. \\

% Figure 3.2
\ntuepsfig[width=\textwidth]{3_2}{An ARM embedded processor}

\noindent Programmable-System-on-a-chip (PSoC) is a hybrid way approach of SoC embedded system. It replaced the traditional ASIC by a Programmable Logic (FPGA) interfacing and embedded multipurpose processor with an integrated Processing system (usually a hard processor) through a high-speed bus. For zynq PSoC, the Advanced eXtensible Interface (AXI) is this high speed communicating bus. The PSoC cannot achieve the high-efficiency as the traditional ASIC approach but thanks to the modern technology and the short development time, the PSoC approach is the most suitable candidate for research and development projects which is below 1000 products for a cycle. \\

% Figure 3.3
\ntuepsfig[width=\textwidth]{3_3}{PS and PL connected through AXI}

\noindent The PSoC board which is used to implement Star Tracking algorithm in this project is the Zynq zedboad illustrated in Figure~\ref{fig:3_4}. The defining feature of Zynq is that it combines a dual-core ARM-A9 processor with traditional FPGA logic fabric\cite{Xilinx}. Although dedicated processors have been coupled with FPGAs before, it has never been quite the same proposition. In Zynq, the ARM Cortex-A9 is an application grade processor, capable of running full operating systems such as Linux, while the programmable logic is based on Xilinx 7-series FPGA architecture. \\

% Figure 3.4
\ntuepsfig[width=\textwidth]{3_4}{A Zedboard Programmable System on Chip}

%%%%%%%%%%%%%%%%%%%%%%%%%%%%%%%%%%%%%%%%%%%%%%%%%%%%%%%%%%%%%%%%%%%%%%%%%%%%%%%%%%%%%%%%%%%%%%%%%%%%%%%%%%%%%%%%%%%%%%%%%%%%%%%%
% Section 3.2
\section{Partition and Profiling}

%%%%%%%%%%%%%%%%%%%%%%%%%%%%
% subsection 3.2.1
\subsection{Partition}

Partitioning is the first stage of PSoC development. The algorithm or the program will be partitioned into suitable-size submodules. These submodules are later decided to implement on which side of the PSoC: Programmable Logics or Processing System. How the program partitioned is decided by the designer. \\

% Figure 3.5
\ntuepsfig[width=\textwidth]{3_5}{Module partitioning\cite{Zynqbook}}

%%%%%%%%%%%%%%%%%%%%%%%%%%%%
% subsection 3.2.2
\subsection{Profiling}

Profiling is a practical analysis that measures and analyze the space and time complexity of an algorithm or a submodule of an algorithm. By examine the profiling result, we can specify the most critical time-consuming part or module of the system. Then based on the profiling and asymptotic analysis stated above, we can decide correctly which parts or submodules can be implemented in hardware instead of software. \\

\noindent All the submodules are implemented on Processing System side by software approach then determined which submodules take the most time to run. These modules then are optimized and designed as Intellectual Property core by Hardware description language (HDL) on Programmable Logics side. \\

% Figure 3.6
\ntuepsfig[width=\textwidth]{3_6}{Profiling steps}

%%%%%%%%%%%%%%%%%%%%%%%%%%%%
% subsection 3.2.3
\subsection{Partitioning and profiling results}

The star tracking algorithm is partitioned into 4 submodules shown in Figure~\ref{fig:3_7}. The 4 submodules include:
\begin{itemize}
	\item \textbf{Centroiding}: Takes the star image and reproduce the star coordinates in the image.
	\item \textbf{Choose the reference star}: From the list of star coordinates, proceeds to choose the reference star.
	\item \textbf{Find the star pattern}: From the reference star, generates the feature vector of the image.
	\item \textbf{Pattern searching}: Search and match the feature vector to the prebuilt tree database to get the starID of the reference star.
\end{itemize}

% Figure 3.7
\ntuepsfig[width=\textwidth]{3_7}{Star tracking algorithm submodules partitioned.}

\noindent After partitioning, all submodules are implemented on the Processing system side to examine the runtime of each submodules. The result of profiling stage in Figure~\ref{fig:3_8}. \\

% Figure 3.8
\ntuepsfig[width=\textwidth]{3_8}{Profiling result.}

% Table 3.1
\noindent Table~\ref{tab:profiling}.

\begin{ntutab}{|c|c|}{profiling}{Module Profiling Result}
	\hline
	\textbf{Modules} & \textbf{Runtime(nanoseconds)} \\
	\hline
	Centroiding & 17688 \\
	\hline
	Choose Reference Star & 54 \\
	\hline
	Find the star pattern & 30 \\
	\hline
	Pattern matching (with tree database) & 20 \\
	\hline
\end{ntutab}

\noindent Apparently, the most time-consuming submodule is the centroiding module. It can be explained because the centroiding stage is an image processing stage. Implementation of an image processing algorithm on general purpose processors is more costly and less effective than any other parallel processing approach like using a graphics processing unit (GPU), in this thesis, we design an IP core in the Programmable Logic side instead.

%%%%%%%%%%%%%%%%%%%%%%%%%%%%%%%%%%%%%%%%%%%%%%%%%%%%%%%%%%%%%%%%%%%%%%%%%%%%%%%%%%%%%%%%%%%%%%%%%%%%%%%%%%%%%%%%%%%%%%%%%%%%%%%%
% Section 3.3
\section{Module implementation}

%%%%%%%%%%%%%%%%%%%%%%%%%%%%
% subsection 3.3.1
\subsection{Centroiding}

%%%%%%%%%
% subsubsection 3.3.1.1
\subsubsection{Thresholding}

In the first stage, the sky is captured by a Charge-coupled device(CCD) through lens then discretized into digital star images. In which, Each pixel is an 8-bit integer value which represents the intensity or brightness of a pixel(0 represent darkest pixel and 255 represents the brightest pixel). The star image is usually contaminated with noise as a result of dark current noise and non-star stellar object. Dark current noise arises from thermal energy within the silicon lattice. It can be represented as a background image generated by a certain period of exposure with the shutter closed\cite{MDP}. To eliminate the noise and separate star clusters from the background, we apply thresholding technique. The threshold level can be chosen by averaging all pixel intensities. The pixel which has a higher value than threshold level belongs to a star cluster while the pixel which has a lower value than the threshold belongs to the background. Figure~\ref{fig:3_10} shows how to separate star clusters from the background, the threshold, in this case, is 70.

\ntuepsfig[scale=0.7]{3_10}{Thresholding separates star cluster pixels and background pixels}


%%%%%%%%%
% subsubsection 3.3.1.2
\subsubsection{The connected-components problem in centroiding}

The raw image is contaminated with noises and other non-stellar objects. These noises can be canceled by setting a threshold level. If a pixel Intensity is less than the threshold level, we will set its value to 0(the background level) to differentiate from star cluster pattern. After all false star noises are canceled, the centroiding algorithm proceeds. \\

\noindent The technique to determine star centroids in a star image employs the connected component finding. The traditional approach of connected component finding is considering each pixel in the star image as a vertex in a graph. If a group of pixels belongs to a star cluster pattern, those vertices are connected. Therefore we can determine each independent star clusters by using graph traversing technique like Depth-First-Search (DFS) or Bread-First-Search (BFS). However, this submodule of the star tracking algorithm is implemented as an IP core by Hardware Description Language (HDL) which lack abilities to implement a recursive function or advanced data structure like Queue, Stack. \\

\noindent The emergence of FPGAs with enough capacity to perform complex image processing tasks also led to high-performance architectures for connected component labeling\cite{CCA}. Most of these architectures utilize the one-pass scan algorithm, because of the limited memory resources available on an FPGA. These types of connected component labeling architectures are able to process several image pixels in parallel, thereby enabling a high throughput at low processing latency to be achieved\cite{000384075100011n.d.}. \\

% Figure 3.9
\ntuepsfig[scale=0.2]{3_9}{Star clusters represented as a connected-components problem}

\noindent Figure~\ref{fig:3_9} represents the star centroid finding problem as finding a connected-components problem. The same star cluster pixels are neighboring to each other. To determine the centroid of each star cluster, we must label the pixels into their corresponding star clusters(groups) then calculate each star cluster centroid of each cluster. \\

%%%%%%%%%
% subsubsection 3.3.1.3
\subsubsection{The one-pass scan algorithm}

The one-pass scan algorithm is a connected component labeling approach which performs a scanning throughout the star image. A pixel in a star cluster will be put into an equivalent independent subset if two subsets having pixels connected to each other will be merged into one big set by the applying the following rules. \\

\noindent We maintain a disjoint set data structure when scanning performed if a pixel P is non-background, we consider its left and above pixels as in Figure~\ref{fig:3_11}. There are 3 cases:

\begin{enumerate}
	\item \textbf{Case 1 - Both are background pixels}: Label the pixel (P) with a new number. Create a new subset.
	\item \textbf{Case 2 - One(either pixel) is labeled, the other is not or both are labeled with the same number}: Label the pixel (P) with this number.
	\item \textbf{Case 3 - Both are labeled with different numbers}: Label the pixel (P) with any number from the left or above pixels, merge the set of the above pixel into the set of the left pixel.
\end{enumerate}

% Figure 3.11
\ntuepsfig[scale=0.2]{3_11}{Left and above pixels of the current considering pixel.}

\noindent An example of the one-pass scan algorithm applied to label 2-star clusters: In Figure~\ref{fig:3_12_2}, the first non-background pixel P has been scanned and comes into case 1(both left and above pixels are background pixels), so we create a new disjoint set(set 1) and add this pixel to set 1. 

% Figure 3.10
\ntuepsfig[scale=0.35]{3_12_2}{one-pass scan algorithm applied to label 2-star clusters - 1}

\noindent Next, we scan the pixel in Figure~\ref{fig:3_12_3}, the above pixel is a background pixel, and its left pixel is in set 1. Therefore, we apply case 2 of the algorithm, set this pixel to the same set as its left pixel which is set 1.

\ntuepsfig[scale=0.35]{3_12_3}{one-pass scan algorithm applied to label 2-star clusters - 2}

\noindent After skipping background pixels, we consider the pixel shown in Figure~\ref{fig:3_12_4}, both above and left pixels of this pixel are background pixels(case 1). We create a new disjoint set(set 2) and add this pixel to set 2. 

\ntuepsfig[scale=0.35]{3_12_4}{one-pass scan algorithm applied to label 2-star clusters - 3}

\noindent The next pixel in Figure~\ref{fig:3_12_5} belongs to case 3, so we merge 2 disjoint sets 1 and 2 and label this pixel as belonging to set 2. 

\ntuepsfig[scale=0.35]{3_12_5}{one-pass scan algorithm applied to label 2-star clusters - 4}

\noindent At the final stage in Figure~\ref{fig:3_12_26}, we successfully find the 2 connected-component which are the pixels belongs to set 1,2, and the pixels belongs to set 3,4.

\ntuepsfig[scale=0.35]{3_12_26}{one-pass scan algorithm applied to label 2-star clusters - 5}

\noindent After detecting the star clusters, we calculate the centroids of each star cluster by equation~\ref{eq:3_1}

% Equation 3.1
\begin{equation}
	x = \frac{\displaystyle\sum_{i=1}^{n} \displaystyle\sum_{j=1}^{m} (i * I_{ij})} {\displaystyle\sum_{i=1}^{n} \displaystyle\sum_{j=1}^{m} I_{ij}}, y = \frac{\displaystyle\sum_{i=1}^{n} \displaystyle\sum_{j=1}^{m} (j * I_{ij})} {\displaystyle\sum_{i=1}^{n} \displaystyle\sum_{j=1}^{m} I_{ij}}
	\label{eq:3_1}
\end{equation}

\noindent The reason to choose the one-pass scan algorithm is that an image can not be loaded into the hardware processing unit as a matrix or tensor but as a pixel stream. We only know the previous pixels of the current receiving pixel.

%%%%%%%%%
% subsubsection 3.3.1.4
\subsubsection{IP core design for the one-pass scan algorithm}

\textbf{An image as streamed pixels} \\
\noindent From the one-pass scan algorithm, we need to design an IP core for faster hardware processing. The input of the module is a stream of pixels, and the output is the centroid coordinates of the stars in this image as shown in Figure~\ref{fig:3_18}.

\ntuepsfig[width=\textwidth]{3_18}{IP core design purpose}

\noindent In figure~\ref{fig:3_17}, the pixel coordinates(i,j) in 2D images would be represented by the stream order(k). The reason we need to have a converting framework is that we need to look up the above and left pixels of a considering pixel in the one-pass scan algorithm. \\
	
\ntuepsfig[width=\textwidth]{3_17}{Stream an image into a pixel stream}

\noindent The converting equations:
	$$k = i*ImgHeight + j$$
	$$i = k / ImgHeight$$
	$$j = k mod ImgHeight$$

\textbf{IP core submodules} \\

\noindent The centroiding again is partitioned into submodules for specific purposes as Figure~\ref{fig:3_19}.

\ntuepsfig[width=\textwidth]{3_19}{IP core submodules}

\begin{itemize}
	\item \textbf{stream processing}: The submodule task is to process the one-pass scan algorithm after receiving a pixel from the stream. The submodule also in responsibility of maintaining and updating the labels and disjoint sets data structures stored in the BRAM. The pseudo codes of the implementation are shown in Figure~\ref{fig:3_20} and Figure~\ref{fig:3_21}
	\item \textbf{centroid calculating}: The submodule task is to calculate the centroid coordinates after the stream processing stage is completed based on Equation~\ref{eq:2_1}. The inputs of the centroid calculating submodule is the disjoint sets data structures. The pseudo codes of the implementation are shown in Figure~\ref{fig:3_22}
	\item \textbf{BRAM}: served as the memory for the IP core. Stored the temporal data structures like: labels and disjoint sets and the final results(star centroid list).
\end{itemize}

\ntuepsfig[scale=0.6]{3_20}{Get above and left pixel functions}

\ntuepsfig[scale=0.6]{3_21}{Stream processing submodule}

\ntuepsfig[scale=0.6]{3_22}{centroid calculating submodule}

\ntuepsfig[scale=0.6]{3_23}{The complete IP core module}

%%%%%%%%%%%%%%%%%%%%%%%%%%%%
% subsection 3.3.2
\subsection{Choose the reference star}

Figure~\ref{fig:3_13} shows a basic star image which then will be processed through the “Choose the reference star” stage. The first Image in an example is a typical star image. The outer circle represents the FOV of the star Image, and the small blue dots represent the star clusters. 

\ntuepsfig[scale=0.35]{3_13}{Distances from the centre of the image to all stars}

\noindent From the centre of the star Image, we calculate the distances from each star cluster to the centre by Euclidean distance as equation~\ref{eq:3_2}. The star which is nearest to the centre (have the minimum distance value) will be chosen as the reference star.

% Equation 3.2
\begin{equation}
	d_{centre\_to\_star} = \sqrt{(x_{star} - x_{centre})^2 + (y_{star} - y_{centre})^2}
	\label{eq:3_2}
\end{equation}

%%%%%%%%%%%%%%%%%%%%%%%%%%%%
% subsection 3.3.3
\subsection{Find the star pattern}

After we have the reference star, the next stage is to generate the star pattern. The number of stars in the image is counted, and their distances to the reference star are calculated as equation~\ref{eq:3_3} to form a feature vector $f = \{N, D_1, D_2, D_3, … D_n\}$

% Equation 3.3
\begin{equation}
	d_{centroid\_to\_star} = \sqrt{(x_{star} - x_{centroid})^2 + (y_{star} - y_{centroid})^2}
	\label{eq:3_3}
\end{equation}

\ntuepsfig[scale=0.35]{3_14}{Distances from the reference star to its neighbors}

%%%%%%%%%%%%%%%%%%%%%%%%%%%%
% subsection 3.3.4
\subsection{Pattern searching}

In the last stage, the feature vector is matched to the prebuilt tree pattern database. The feature vector from the “Generate feature vector stage” will be compared to the prebuilt tree structured star pattern database(SPD) to find the correct star cluster ID. Figure~\ref{fig:3_15} shows how a star vector is compared and match with the SPD. \\

\ntuepsfig[width=\textwidth]{3_15}{Star pattern matching}

\noindent In case of mismatching, a tolerance will be added to research for another star ID. This search can be performed parallel to increase the searching speed due to the specific tree structure of the prebuilt SPD as in Figure~\ref{fig:3_16}.

\ntuepsfig[width=\textwidth]{3_16}{Star pattern matching with tolerance}
