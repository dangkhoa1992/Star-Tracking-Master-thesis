% Appendix D
\chapter{Computation of Geometrical Bounds in a Multi-camera Setup} % Main appendix title
\label{append:multicamera} % For referencing this appendix elsewhere, use \ref{append:motor}



\lhead{Appendix D. \emph{Computation of Geometrical Bounds}} % This is for the header on each page - perhaps a shortened title

We formulate the geometrical limits for other scenarios as well. %, with noise in X co-ordinate of the camera positions.

% Part 1
\section{Noise in X- coordinate}
 
\subsection{Scenario - II}

\subsubsection{Upper Bound}

\begin{minipage}[t]{0.5\textwidth}
\includegraphics[width=0.7\textwidth,bb=0 0 600 600]{X-2a.pdf}
\captionof{figure}{Determination of upper bound for the translation of camera in X-direction (Scenario II).}
\end{minipage}
\begin{minipage}[t]{0.5\textwidth}
\vspace{-2in}
\begin{equation*}
\centering
\begin{aligned}
ZC_x = r\cos(\theta) \\
Z'C_y = r\sin(\theta) \\
ZC_x = \frac{r\sin(\theta)}{\tan(\theta_5)}\\
Z'Z = \frac{r\sin(\theta)}{\tan(\theta_5)} - r\cos(\theta)\\
\end{aligned}
\end{equation*}
\end{minipage}






\subsubsection{Lower Bound}

\begin{minipage}[t]{0.5\textwidth}
\includegraphics[width=0.9\textwidth,bb=0 0 600 600]{X-2b.pdf} 
\captionof{figure}{Determination of lower bound for the translation of camera in X-direction (Scenario II).}
\end{minipage}
\begin{minipage}[t]{0.5\textwidth}
\vspace{-2in}
\begin{equation*}
\centering
\begin{aligned}
ZC_x = r\cos(\theta) \\
ZC_y =  C_{x}C = r\sin(\theta) \\
Z'C_x =\frac{r\sin(\theta)}{\tan(\theta_4)}\\
Z'Z = \frac{r\sin(\theta)}{\tan(\theta_4)} - r\cos(\theta)\\
\end{aligned}
\end{equation*}
\end{minipage}


Therefore, the two geometrical bounds - lower and upper are $-(\frac{r\sin(\theta)}{\tan(\theta_4)} - r\cos(\theta))$ and $-(\frac{r\sin(\theta)}{\tan(\theta_5)} - r\cos(\theta))$ respectively.


\subsection{Scenario - III}

\subsubsection{Upper Bound}

\begin{minipage}[t]{0.5\textwidth}
\includegraphics[width=0.65\textwidth,bb=0 0 600 600]{X-3a.pdf} 
\captionof{figure}{Determination of upper bound for the translation of camera in X-direction (Scenario III).}
\end{minipage}
\begin{minipage}[t]{0.5\textwidth}
\vspace{-2in}
\begin{equation*}
\centering
\begin{aligned}
\sin(180-\theta) = \frac{CC_x}{CZ}\\
\implies CC_x = r\sin(\theta) \\ \\
\tan(\theta_5) = \frac{CC_x}{Z'C_x}\\
\implies Z'C_x = \frac{r\sin(\theta)}{\tan(\theta_5)}\\\\
\cos(180-\theta) = \frac{C_{x}Z}{CZ}\\
\implies C_{x}Z = -r\cos(\theta)\\ \\
Z'Z = \frac{r\sin(\theta)}{\tan(\theta_5)} + r\cos(\theta)
\end{aligned}
\end{equation*}
\end{minipage}




\subsubsection{Lower Bound}

\begin{minipage}[t]{0.5\textwidth}
\includegraphics[width=0.65\textwidth,bb=0 0 600 600]{X-3b.pdf} 
\captionof{figure}{Determination of upper bound for the translation of camera in X-direction (Scenario III).}
\end{minipage}
\begin{minipage}[t]{0.5\textwidth}
\vspace{-2in}
\begin{equation*}
\centering
\begin{aligned}
\sin(180-\theta) = \frac{CC_x}{CZ}\\
\implies CC_x = r\sin(\theta) \\ \\
\tan(\theta_4) = \frac{CC_x}{Z'C_x}\\
\implies Z'C_x = \frac{r\sin(\theta)}{\tan(\theta_4)}\\\\
\cos(180-\theta) = \frac{C_{x}Z}{CZ}\\
\implies C_{x}Z = -r\cos(\theta)\\ \\
Z'Z = \frac{r\sin(\theta)}{\tan(\theta_4)} + r\cos(\theta)
\end{aligned}
\end{equation*}
\end{minipage}

Therefore, the two geometrical limits - lower and upper are given by $-(\frac{r\sin(\theta)}{\tan(\theta_4)} + r\cos(\theta))$ and $-(\frac{r\sin(\theta)}{\tan(\theta_5)} + r\cos(\theta))$ respectively.

\subsection{Scenario - IV}
In this scenario, \emph{no} possible translation of camera in X-direction can contain the point in the active region. Therefore, the probability of a point to belong to active region is \emph{null}.

\begin{figure}[H]
\centering
\includegraphics[width=0.45\textwidth,bb=0 0 600 600]{X-4.pdf} 
\caption{Probability is $0$ for the translation of camera in X-direction (Scenario IV).}
\label{fig:SWIM-images}
\end{figure}


\subsection{Scenario - V}
In this scenario, \emph{no} possible translation of camera in X-direction can contain the point in the active region. Therefore, the probability of a point to belong to active region is \emph{null}.

\begin{figure}[htb]
\centering
\includegraphics[width=0.45\textwidth,bb=0 0 600 600]{X-5.pdf} 
\caption{Probability is $0$ for the translation of camera in X-direction (Scenario V).}
\label{fig:SWIM-images}
\end{figure}



\subsection{Scenario - VI}

\begin{minipage}[t]{0.5\textwidth}
\includegraphics[width=0.65\textwidth,bb=0 0 600 600]{X-6a.pdf}
\captionof{figure}{Determination of lower bound for the translation of camera in X-direction (Scenario VI).}
\end{minipage}
\begin{minipage}[t]{0.5\textwidth}
\vspace{-2in}
\begin{equation*}
\centering
\begin{aligned}
ZC_x = r\cos(\theta) \\
ZC_y = r\sin(\theta) \\
Z'C_x = \frac{r\sin(\theta)}{\tan(\theta_4)}\\
ZZ' = r\cos(\theta) - \frac{r\sin(\theta)}{\tan(\theta_4)}
\end{aligned}
\end{equation*}
\end{minipage}





\subsubsection{Upper Bound}

\begin{minipage}[t]{0.5\textwidth}
\includegraphics[width=0.65\textwidth,bb=0 0 600 600]{X-6b.pdf} 
\captionof{figure}{Determination of upper bound for the translation of camera in X-direction (Scenario VI).}
\end{minipage}
\begin{minipage}[t]{0.5\textwidth}
\vspace{-2in}
\begin{equation*}
\centering
\begin{aligned}
ZC_x = r\cos(\theta) \\
ZC_y = r\sin(\theta) \\
Z'C_x = \frac{r\sin(\theta)}{\tan(\theta_5)}\\
ZZ' = r\cos(\theta) - \frac{r\sin(\theta)}{\tan(\theta_5)}
\end{aligned}
\end{equation*}
\end{minipage}


Therefore, the two geometrical limits - lower and upper are $+(r\cos(\theta) - \frac{r\sin(\theta)}{\tan(\theta_4)})$ and $r\cos(\theta) - \frac{r\sin(\theta)}{\tan(\theta_5)}$ respectively.


% Part 2
\section{Noise in Y- coordinate}

\subsection{Scenario - II}

\subsubsection{Lower Bound}

\begin{minipage}[t]{0.5\textwidth}
\includegraphics[width=0.65\textwidth,bb=0 0 600 600]{Y-2a.pdf}  
\captionof{figure}{Determination of lower bound for the translation of camera in Y-direction (Scenario II).}
\end{minipage}
\begin{minipage}[t]{0.5\textwidth}
\vspace{-2in}
\begin{equation*}
\centering
\begin{aligned}
\cos(90-\theta)=\frac{C_{y}Z}{CZ}\\
\implies C_{y}Z=r\sin(\theta)\\
\sin(90-\theta) = \frac{C_{y}C}{CZ}\\
\implies C_{y}C = r\cos(\theta)\\
\tan(90-\theta_5) = \frac{C_{y}C}{C_{y}Z'}\\
\implies C_{y}Z' = \frac{r\cos(\theta)}{\cot(\theta_5)}\\
ZZ' =  r\sin(\theta) - \frac{r\cos(\theta)}{\cot(\theta_5)}
\end{aligned}
\end{equation*}
\end{minipage}





\subsubsection{Upper Bound}

\begin{minipage}[t]{0.5\textwidth}
\includegraphics[width=0.65\textwidth,bb=0 0 600 600]{Y-2b.pdf}  
\captionof{figure}{Determination of upper bound for the translation of camera in Y-direction (Scenario II).}
\end{minipage}
\begin{minipage}[t]{0.5\textwidth}
\vspace{-2in}
\begin{equation*}
\centering
\begin{aligned}
\cos(90-\theta)=\frac{C_{y}P}{CP}\\
\implies C_{y}P=r\sin(\theta)\\
\sin(90-\theta) = \frac{C_{y}C}{CP}\\
\implies C_{y}C = r\cos(\theta)\\
\tan(90-\theta_4) = \frac{C_{y}C}{C_{y}P'}\\
\implies C_{y}P' = \frac{r\cos(\theta)}{\cot(\theta_4)}\\
PP' =  r\sin(\theta) - \frac{r\cos(\theta)}{\cot(\theta_4)}
\end{aligned}
\end{equation*}
\end{minipage}







\subsection{Scenario - III}

\begin{figure}[H]
	\centering
	\includegraphics[width=0.4\textwidth,bb=0 0 600 600]{Y-3.pdf} 
	\caption{Probability of the point C is $0$ for the translation of camera in Y-direction (Scenario III).}
	\label{fig:SWIM-images}
\end{figure}


\subsection{Scenario - IV}

\begin{figure}[H]
	\centering
	\includegraphics[width=0.4\textwidth,bb=0 0 600 600]{Y-4.pdf} 
	\caption{Probability of the point C is $0$ for the translation of camera in Y-direction (Scenario IV).}
	\label{fig:SWIM-images}
\end{figure}


\subsection{Scenario - V}

\subsubsection{Lower Bound}

\begin{minipage}[t]{0.5\textwidth}
\includegraphics[width=0.55\textwidth,bb=0 0 600 600]{Y-5a.pdf}  
\captionof{figure}{Determination of lower bound for the translation of camera in Y-direction (Scenario V).}
\end{minipage}
\begin{minipage}[t]{0.5\textwidth}
\vspace{-2in}
\begin{equation*}
\centering
\begin{aligned}
\alpha = 90 - (360 - \theta)\\
\implies \alpha = \theta - 270 \\
\sin(\alpha) = \frac{CC_y}{CP}\\
CC_y = r\sin(\theta - 270)\\
\cos(\alpha) = \frac{PC_y}{CP}\\
\implies PC_y = r\cos(\theta - 270)\\
\tan(90 - \theta_5) = \frac{CC_y}{C_{y}P'}\\
\implies C_{y}P' = \frac{r\sin(\theta - 270)}{\cot(\theta_4)}\\
PP' = r\cos(\theta - 270) +  \frac{r\sin(\theta - 270)}{\cot(\theta_4)}
\end{aligned}
\end{equation*}
\end{minipage}







\subsubsection{Upper Bound}

\begin{minipage}[t]{0.5\textwidth}
\includegraphics[width=0.55\textwidth,bb=0 0 600 600]{Y-5b.pdf}  
\captionof{figure}{Determination of upper bound for the translation of camera in Y-direction (Scenario V).}
\end{minipage}
\begin{minipage}[t]{0.5\textwidth}
\vspace{-2in}
\begin{equation*}
\centering
\begin{aligned}
\alpha = 90 - (360 - \theta)\\
\implies \alpha = \theta - 270 \\
\sin(\alpha) = \frac{CC_y}{CP}\\
CC_y = r\sin(\theta - 270)\\
\cos(\alpha) = \frac{PC_y}{CP}\\
\implies PC_y = r\cos(\theta - 270)\\
\tan(90 - \theta_4) = \frac{CC_y}{C_{y}P'}\\
\implies C_{y}P' = \frac{r\sin(\theta - 270)}{\cot(\theta_4)}\\
PP' = r\cos(\theta - 270) +  \frac{r\sin(\theta - 270)}{\cot(\theta_4)}
\end{aligned}
\end{equation*}
\end{minipage}






\subsection{Scenario - VI}

\subsubsection{Upper Bound}

\begin{minipage}[t]{0.5\textwidth}
\includegraphics[width=0.55\textwidth,bb=0 0 600 600]{Y-6a.pdf}  
\captionof{figure}{Determination of upper bound for the translation of camera in Y-direction (Scenario VI).}
\end{minipage}
\begin{minipage}[t]{0.5\textwidth}
\vspace{-2in}
\begin{equation*}
\centering
\begin{aligned}
\cos(90-\theta)=\frac{C_{y}P}{CP}\\
\implies C_{y}P=r\sin(\theta)\\
\sin(90-\theta) = \frac{C_{y}C}{CP}\\
\implies C_{y}C = r\cos(\theta)\\
\tan(90-\theta_4) = \frac{C_{y}C}{C_{y}P'}\\
\implies C_{y}P' = \frac{r\cos(\theta)}{\cot(\theta_4)}\\
PP' = \frac{r\cos(\theta)}{\cot(\theta_4)} - r\sin(\theta)
\end{aligned}
\end{equation*}
\end{minipage}







\subsubsection{Lower Bound}

\begin{minipage}[t]{0.5\textwidth}
\includegraphics[width=0.75\textwidth]{demo3.pdf}   
\captionof{figure}{Determination of lower bound for the translation of camera in Y-direction (Scenario VI).}
\end{minipage}
\begin{minipage}[t]{0.5\textwidth}
\vspace{-2in}
\begin{equation*}
\centering
\begin{aligned}
\cos(90-\theta)=\frac{C_{y}P}{CP}\\
\implies C_{y}P=r\sin(\theta)\\
\sin(90-\theta) = \frac{C_{y}C}{CP}\\
\implies C_{y}C = r\cos(\theta)\\
\tan(90-\theta_5) = \frac{C_{y}C}{C_{y}P'}\\
\implies C_{y}P' = \frac{r\cos(\theta)}{\cot(\theta_5)}\\
PP' = \frac{r\cos(\theta)}{\cot(\theta_5)} - r\sin(\theta)
\end{aligned}
\end{equation*}
\end{minipage}








% Part 3
\section{Noise in view angle}

\subsection{Scenario - II}

\subsubsection{Lower Bound}

\begin{minipage}[t]{0.5\textwidth}
\includegraphics[width=0.9\textwidth]{theta-2a.pdf}   
\captionof{figure}{Determination of lower bound of view angle for the rotation of camera (Scenario II).}
\end{minipage}
\begin{minipage}[t]{0.5\textwidth}
\vspace{-1in}
\begin{equation*}
\centering
\begin{aligned}
\theta_5^{'} = \theta_5 + (\theta - \theta_5)\\
\Theta + \delta_5 = \theta \\
\implies \Theta = \theta - \delta_5
\end{aligned}
\end{equation*}
\end{minipage}









\subsubsection{Upper Bound}

\begin{minipage}[t]{0.5\textwidth}
\includegraphics[width=0.85\textwidth]{theta-2b.pdf}   
\captionof{figure}{Determination of upper bound of view angle for the rotation of camera (Scenario II).}
\end{minipage}
\begin{minipage}[t]{0.5\textwidth}
\vspace{-1in}
\begin{equation*}
\centering
\begin{aligned}
\theta_5^{'} = \theta_5 + (\theta - \theta_4)\\
\Theta + \delta_5 = \theta_5 + (\theta - \theta_4)\\
\implies \Theta = \theta_5 + (\theta - \theta_4) - \delta_5
\end{aligned}
\end{equation*}
\end{minipage}






\subsection{Scenario - III}

\subsubsection{Lower Bound}

\begin{minipage}[t]{0.5\textwidth}
\includegraphics[width=0.85\textwidth]{theta-3a.pdf}  
\captionof{figure}{Determination of lower bound of view angle for the rotation of camera (Scenario III).}
\end{minipage}
\begin{minipage}[t]{0.5\textwidth}
\vspace{-1in}
\begin{equation*}
\centering
\begin{aligned}
\theta_5^{'} = \theta_5 + (\theta - \theta_5)\\
\Theta + \delta_5 = \theta \\
\implies \Theta = \theta - \delta_5
\end{aligned}
\end{equation*}
\end{minipage}






\subsubsection{Upper Bound}

\begin{minipage}[t]{0.5\textwidth}
\includegraphics[width=0.85\textwidth]{theta-3b.pdf}  
\captionof{figure}{Determination of upper bound of view angle for the rotation of camera (Scenario III).}
\end{minipage}
\begin{minipage}[t]{0.5\textwidth}
\vspace{-1in}
\begin{equation*}
\centering
\begin{aligned}
\theta_5^{'} = \theta_5 + (\theta - \theta_4)\\
\Theta + \delta_5 = \theta_5 + (\theta - \theta_4)\\
\implies \Theta = \theta_5 + (\theta - \theta_4) - \delta_5
\end{aligned}
\end{equation*}
\end{minipage}









\subsection{Scenario - IV}

\begin{figure}[H]
	\centering
	\includegraphics[width=0.4\textwidth]{theta-4.pdf} 
	\caption{Probability of the point C is $0$ for the rotation of camera (Scenario IV).}
	\label{fig:SWIM-images}
\end{figure}



\subsection{Scenario - V}

\subsubsection{Upper Bound}

\begin{minipage}[t]{0.5\textwidth}
\includegraphics[width=0.85\textwidth]{theta-5a.pdf} 
\captionof{figure}{Determination of upper bound of view angle for the rotation of camera (Scenario V).}
\end{minipage}
\begin{minipage}[t]{0.5\textwidth}
\vspace{-1in}
\begin{equation*}
\centering
\begin{aligned}
\theta_4^{'} = \theta_4 - (\theta_4 + 360 -\theta)\\
\implies \theta_4^{'} = \theta - 360 \\
\Theta + \delta_4 = \theta - 360 \\
\implies \Theta = \theta - 360 - \delta_4
\end{aligned}
\end{equation*}
\end{minipage}







\subsubsection{Lower Bound}

\begin{minipage}[t]{0.5\textwidth}
\includegraphics[width=0.85\textwidth]{theta-5b.pdf} 
\captionof{figure}{Determination of lower bound of view angle for the rotation of camera (Scenario V).}
\end{minipage}
\begin{minipage}[t]{0.5\textwidth}
\vspace{-1in}
\begin{equation*}
\centering
\begin{aligned}
\theta_4^{'} = \theta_4 - (\theta_5 + 360 -\theta)\\
\implies \theta_4^{'} = \theta_4 - \theta_5 - 360 + \theta \\
\Theta + \delta_4 = \theta_4 - \theta_5 - 360 + \theta \\
\implies \Theta = \theta_4 - \theta_5 - 360 + \theta - \delta_4 
\end{aligned}
\end{equation*}
\end{minipage}








\subsection{Scenario - VI}

\subsubsection{Upper Bound}

\begin{minipage}[t]{0.5\textwidth}
\includegraphics[width=0.85\textwidth]{theta-6a.pdf}  
\captionof{figure}{Determination of upper bound of view angle for the rotation of camera (Scenario VI).}
\end{minipage}
\begin{minipage}[t]{0.5\textwidth}
\vspace{-1in}
\begin{equation*}
\centering
\begin{aligned}
\theta_4^{'} = \theta_4 - (\theta_4 - \theta)\\
\Theta + \delta_4 = \theta \\
\implies \Theta = \theta - \delta_4
\end{aligned}
\end{equation*}
\end{minipage}








\subsubsection{Lower Bound}

\begin{minipage}[t]{0.5\textwidth}
\includegraphics[width=0.85\textwidth]{theta-6b.pdf}  
\captionof{figure}{Determination of lower bound of view angle for the rotation of camera (Scenario VI).}
\end{minipage}
\begin{minipage}[t]{0.5\textwidth}
\vspace{-1in}
\begin{equation*}
\centering
\begin{aligned}
\theta_4^{'} = \theta_4 - (\theta_5 - \theta)\\
\Theta + \delta_4 = \theta_4 - \theta_5 + \theta\\
\implies \Theta = \theta_4 - \theta_5 + \theta - \delta_4
\end{aligned}
\end{equation*}
\end{minipage}




